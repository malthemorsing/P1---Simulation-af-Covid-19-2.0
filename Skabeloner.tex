
----------------------
SKABELONER OG GODE RÅD
----------------------

I dette dokument er der samlet en masse skabeloner til at kopiere ind i dine LaTeX-filer samt gode råd til brug af LaTeX. Det er opdelt sektionsvis (billeder, tabeller mv.). Med store bogstaver er det angivet, hvor du skal give specifikke input til koden, fx hvilket billede du ønsker at inkludere. Det er ofte fulgt af et eksempel. God fornøjelse!

______________
¤¤ Billeder ¤¤

% Indsaet figurer nemt med:
\figur{STOERRELSE}{FIL}{FIGURTEKST.}{figur:LABEL}

% Enkelt billede:
\begin{figure}[H] % (alternativt [htbp])
	\centering
	\includegraphics[width=STØRRELSE\textwidth]{MAPPE/FIL}
	\caption{FIGURTEKST.}
	\label{figur:LABEL}
\end{figure}

% Eksempel:
\begin{figure}[H]
	\centering
	\includegraphics[width=0.80\textwidth]{billeder/sketch.pdf}
	\caption{Opbygning af model.}
	\label{fig:sketch}
\end{figure}

% 2 billeder side om side:
\begin{figure}[H]
	\centering
	\begin{minipage}[b]{0.48\textwidth}
	\centering
	\includegraphics[width=1.00\textwidth]{MAPPE/FIL1} % Venstre billede
	\end{minipage}
	\hfill
	\begin{minipage}[b]{0.48\textwidth}
	\centering
	\includegraphics[width=1.00\textwidth]{MAPPE/FIL2} % Højre billede
	\end{minipage}
	\\ % Figurtekster og labels
	\begin{minipage}[t]{0.48\textwidth}
	\caption{VENSTRE FIGURTEKST.} % Venstre figurtekst og label
	\label{fig:LABEL1}
	\end{minipage}
	\hfill
	\begin{minipage}[t]{0.48\textwidth}
	\caption{HØJRE FIGURTEKST.} % Højre figurtekst og label
	\label{fig:LABEL2}
	\end{minipage}
\end{figure}

% Eksempel:
\begin{figure}[H]
	\centering
	\begin{minipage}[b]{0.48\textwidth}
	\centering
	\includegraphics[width=1.00\textwidth]{billeder/billede1.jpg} % Venstre billede
	\end{minipage}
	\hfill
	\begin{minipage}[b]{0.48\textwidth}
	\centering
	\includegraphics[width=1.00\textwidth]{billeder/billede2.jpg} % Højre billede
	\end{minipage}
	\\ % Figurtekster og labels
	\begin{minipage}[t]{0.48\textwidth}
	\caption{Den ene figurtekst.} % Venstre figurtekst og label
	\label{fig:billede1}
	\end{minipage}
	\hfill
	\begin{minipage}[t]{0.48\textwidth}
	\caption{Den anden figurtekst.} % Højre figurtekst og label
	\label{fig:billede2}
	\end{minipage}
\end{figure}

% Princippet bag brug af minipages er forsøgt illustreret nedenfor. Koden danner 4 minipages ("bokse"). I de 2 øverste inkluderes billederne, og i de 2 nederste lægges figurtekster og labels. Et linjeskift (\\) adskiller de 2 sæt bokse. I eksemplet ovenfor reserveres 48 pct. af bredden til hver minipage. Det resterende "luft" strækkes mest muligt med \hfill. Inde i hver minipage tillades billedet at fylde 100 pct. af boksens bredde. 

%   |¯¯¯¯|        |¯¯¯¯|
%   |    |        |    |
%   |____| \hfill |____| \\
%   |¯¯¯¯|        |¯¯¯¯|
%   |    |        |    |
%   |____| \hfill |____|

% Princippet kan også benyttes til at have 3 eller flere billeder side om side - tilføj blot flere minipages. Husk at skrue på minipage-størrelsen (fx 0.30 ved 3 billeder), og inkluder to ekstra \hfill. Der skal fortsat kun være ét linjeskift, og billede-størrelsen inde i minipagen skal også fortsat være 100 pct. 

______________
¤¤ Tabeller ¤¤
¯¯¯¯¯¯¯¯¯¯¯¯¯¯

% Eksempel - klassisk tabel:
\begin{table}[H] 
	\centering 
	\begin{tabular}{|l|l|l|l|l|l|} % Afstem antal tegn og kolonner! (l for venstre, c for center, r for højre, | for lodret streg) 
		\hline 	% Vandret streg
					  & Mandag & Tirsdag & Onsdag    & Torsdag   & Fredag  \\ \hline 	% Linjeskift og vandret streg
		09:00 - 10:00 & Kemi   & Dansk   & Matematik & Gymnastik & Engelsk \\ \hline 
		10:00 - 11:00 & Tysk   & Fransk  & Biologi   & Metal     & Fysik   \\ \hline 
	\end{tabular} 
	\caption{Peters skoleskema uge 41.} 
	\label{tab:skoleskema} 
\end{table}

% Eksempel - læsevenlig og flot tabel:
\begin{table}[H]
	\centering
	\begin{tabular}{lccl}	% Afstem antal tegn og kolonner! (l for venstre, c for center, r for højre)
		\toprule
		Case & Bemanding & Rapporter & Noter \\\midrule
		1 & 4 & 13 &            \\
		2 & 3 & 9  & Se notits  \\
		3 & 5 & 12 &            \\
		\bottomrule
	\end{tabular}
	\caption{Overblik over cases.}
	\label{tab:cases}
\end{table}

% Flet kolonner:
\multicolumn{ANTAL}{JUSTERING}{INDHOLD}

% Eksempel:
\multicolumn{5}{c}{Regioner}

_______________
¤¤ Matematik ¤¤
¯¯¯¯¯¯¯¯¯¯¯¯¯¯¯

% Almindelig ligning:
\begin{align}
		
	\label{eq:LABEL} % \nonumber fjerner nummeret (label bliver derved overflødigt)
\end{align}

% Eksempel - med definition af variable efterfølgende:
\begin{align}
	\Phi = \rho \cdot c_p \cdot q_v \cdot \Delta T
	\label{eq:varmeflux}
\end{align}

Hvor:
\begin{table}[H]
	\begin{tabular}{l|l}
	$\Phi$     & Varmestrøm [\si{W}] \\
	$\rho$ 	   & Luftens densitet [\si{kg/m^3}] \\
	$c_p$ 	   & Luftens specifikke varmefylde [\si{J/kgK}] \\
	$q_v$	   & Volumenstrøm [\si{m^3/s}] \\
	$\Delta T$ & Temperaturforskel [\si{K}]
	\end{tabular}
\end{table}

% Eksempel med 2 rækker:
\begin{align}
& x + 2 = 8 	\label{eq:lign1} \\ 	% Rækkerne venstrejusteres ved "&"
& 7 = y + 5 z 	\label{eq:lign2}
\end{align}

% Matematik udenfor aligns (brødtekst, tabeller, figurtekster mv.):
\si{ENHED}
\SI{TAL}{ENHED}
$SPECIALTEGN$

% Eksempler:
\si{m^3}
\SI{9,82}{m/s^2}
$\alpha$

__________
¤¤ Kemi ¤¤
¯¯¯¯¯¯¯¯¯¯

% Kemiske formler:
\ce{FORMEL}

% Eksempler:
\ce{CO2}
\ce{Fe2O3}
\ce{HCO3-}

% RS-sætninger:
\rsphrase{NUMMER}

% Eksempel:
R1: \rsphrase{R1}

____________
¤¤ Labels ¤¤
¯¯¯¯¯¯¯¯¯¯¯¯

% Varianter:
\label{chapter:..}      (Kapitel)
\label{equation:...} 	(Ligning)
\label{figure:...} 	    (Figur)
\label{table:...} 	    (Tabel)

________________________
¤¤ Interne referencer ¤¤
¯¯¯¯¯¯¯¯¯¯¯¯¯¯¯¯¯¯¯¯¯¯¯¯

% Varianter:
\ref{LABEL} 		“4.3”
\vref{LABEL} 		“4.3 on page 31”
\eqref{LABEL} 		“(4.3)”
% På figur \ref{LABEL} ses det at...
% Derudover er det vigtigt at høre efter i timen (figur \ref{LABEL}

_________________________
¤¤ Eksterne referencer ¤¤
¯¯¯¯¯¯¯¯¯¯¯¯¯¯¯¯¯¯¯¯¯¯¯¯¯

% Varianter:
\citep{LABEL} 		(Passiv)
\citet{LABEL} 		(Aktiv)

% Tilføj yderligere information:
\citep[ SIDE, AFSNIT, KAPITEL MV.]{LABEL}

% Eksempler:
\citep[ s. 58]{fysikbog} 		->		[Efternavn, År, s. 58]
\citep[ kap. 7]{fysikbog} 		->		[Efternavn, År, kap. 7]

% NB! Start med et mellemrum i [ ]!

_______________
¤¤ Bilags-CD ¤¤
¯¯¯¯¯¯¯¯¯¯¯¯¯¯¯

% Henvisning til bilags-CD:
\citep[ FILNAVN]{cd}

% I litteratur.bib-filen er der lavet en kilde, som optræder som henvisning til bilags-CD'en. Udskift blot gruppenavn.

_____________
¤¤ Genveje ¤¤
¯¯¯¯¯¯¯¯¯¯¯¯¯

% Genveje kodet nederst i preamble:
$\decC$ 		->		°C 		(ˆ{\circ}\text{C})
$\dec$ 			->		° 		(ˆ{\circ})
\m 				->		· 		(\cdot)

_____________
¤¤ Code Snippets ¤¤
¯¯¯¯¯¯¯¯¯¯¯¯¯
% Multi-line
\begin{lstlisting}[language=c, caption={FIGURTEKST FX: \texttt{funkyName(a, b)}}, captionpos=b, label={snippet:LABELNAVN}]
function funkyName(a, b) {
    const foo = "bar";
    return a + b;
}
\end{lstlisting}

% One-line
\texttt{"Dette er en tekststreng"}
\begin{sloppy}
    \texttt{"Dette er en tekststreng som bryder marginen"} %Hvis margin brydes
\end{sloppy}
