\documentclass[a4paper,12pt,fleqn,twoside,openany]{memoir} 	% Openright aabner kapitler paa hoejresider (openany = vilkaarlig/begge)

%%%% MANUEL OPSAETNING %%%%
\newcommand{\name}{SW1A325a}                 % Skriv navnet på dig eller din gruppe her
\newcommand{\sidefod}{\name}                % Skriv hvad der skal staa til venstre i sidefoden her
%Derudover: kig under 'Taelle ord'

%%%% PAKKER %%%%

\usepackage{url}
\usepackage{svg}
\usepackage{dirtytalk}
\usepackage[ruled]{algorithm2e}
\usepackage{subcaption}

% ¤¤ Oversaettelse og tegnsaetning ¤¤ %
\usepackage[T1,OT1]{fontenc}					% Output-indkodning af tegnsaet, dvs. printede fonte og tegn (T1 = Type 1 font med support for de fleste europaeiske sprog)
\usepackage[english]{babel}					% Sproglig fremstilling af elementer (figur vs. figure, litteratur vs. bibliography osv.)
\usepackage{varioref}
\addto\extrasenglish{% page 5 of varioref's manual
  \renewcommand\reftextfaceafter{on the following page}%
  \renewcommand\reftextafter {on the next page}%
  \renewcommand\reftextfacebefore{on the previous page}%
  \renewcommand\reftextbefore {on the previous page}%
}

\usepackage{ragged2e,anyfontsize}			% Justering af elementer

%Indentation
\usepackage{parskip}										
										
% ¤¤ Figurer og tabeller (floats) ¤¤ %
\usepackage{graphicx} 						% Inkludering af eksterne billeder (JPG, PNG, PDF)
\usepackage{multirow}                		% Fletning af raekker og kolonner (\multicolumn og \multirow)
\usepackage{colortbl} 						% Farver i tabeller (fx \columncolor, \rowcolor og \cellcolor)
\usepackage[dvipsnames]{xcolor}				% Definer farver med \definecolor. Se mere: http://en.wikibooks.org/wiki/LaTeX/Colors
\usepackage{flafter}% Soerger for, at floats ikke optraeder i teksten foer deres reference
\usepackage{float}							% Muliggoer eksakt placering af floats, fx \begin{figure}[H]
\let\newfloat\relax 						% Justering mellem float-pakken og memoir
%\usepackage{eso-pic}						% Tilfoej billedekommandoer paa hver side
%\usepackage{wrapfig}						% Indsaettelse af figurer omsvoebt af tekst 
%\usepackage{multicol}         	        	% Muliggoer tekst i spalter
%\usepackage{rotating}						% Rotation af tekst med \begin{sideways}...\end{sideways}

% ¤¤ Matematik mm. ¤¤
\usepackage{amsmath,amssymb,stmaryrd} 		% Avancerede matematik-udvidelser
\usepackage{mathtools}						% Andre matematik- og tegnudvidelser
\usepackage{textcomp}                 		% Symbol-udvidelser (fx promille-tegn med \textperthousand)
\usepackage{siunitx}						% Flot og konsistent praesentation af tal og enheder med \si{enhed} og \SI{tal}{enhed}
\sisetup{output-decimal-marker = {,}}		% Opsaetning af \SI og decimalseparator
%\usepackage[version=3]{mhchem} 			% Kemi-pakke til flot og let notation af formler, fx \ce{Fe2O3}
%\usepackage{rsphrase}						% Kemi-pakke til RS-saetninger, fx \rsphrase{R1}

% ¤¤ Referencer og kilder ¤¤ %
\usepackage{lastpage}                       % Muliggoer bl.a. brug af \pageref{LastPage} kommandoen
\usepackage[english]{varioref}				% Muliggoer bl.a. krydshenvisninger med sidetal (\vref)
\usepackage{natbib}							% Udvidelse med naturvidenskabelige citationsmodeller, herunder Harvard-modellen
%\usepackage{xr}							% Referencer til eksternt dokument med \externaldocument{<NAVN>}
%\usepackage{glossaries}					% Terminologi- eller symbolliste (se mere i Lars Madsens Latex-bog)
\usepackage[font=itshape]{quoting}                      % Muliggør at indsætte citater med \begin{quoting}...\end{quoting}

% ¤¤ Misc. ¤¤ %
\usepackage{listings} % Placer kildekode i dokumentet med \begin{lstlisting}...\end{lstlisting}

\usepackage{lipsum}							% Dummy tekst med fx \lipsum[2]
\usepackage[shortlabels]{enumitem}			% Muliggoer enkelt konfiguration af lister (se \setlist nedenfor)
\usepackage{pdfpages}						% Goer det muligt at inkludere pdf-dokumenter med kommandoen \includepdf[pages={x-y}]{fil.pdf}	
\pdfoptionpdfminorversion=6					% Muliggoer inkludering af pdf-dokumenter af version 1.6 og hoejere
\pretolerance=2500 							% Justering af afstand mellem ord (hoejt tal, mindre orddeling og mere luft mellem ord)

% Kommentarer og rettelser med \fxnote. Med 'final' i stedet for 'draft' udloeser hver note en error i den faerdige rapport.
\usepackage[footnote,draft,english,silent,nomargin]{fixme}		

% Muliggør blokkommentarer med \begin{comment} ... \end{comment}
\usepackage{comment}

%%%% BRUGERDEFINEREDE INDSTILLINGER %%%%

% ¤¤ Marginer ¤¤ %
\setlrmarginsandblock{2.5cm}{2.5cm}{*}		% \setlrmarginsandblock{Indbinding}{Kant}{Ratio}
\setulmarginsandblock{2.5cm}{2.5cm}{*}		% \setulmarginsandblock{Top}{Bund}{Ratio}
\checkandfixthelayout 						% Oversaetter vaerdier til brug for andre pakker

%	¤¤ Afsnitsformatering ¤¤ %
\setlength{\parindent}{0mm}           		% Stoerrelse af indryk
\setlength{\parskip}{3mm}          			% Afstand mellem afsnit ved brug af double Enter
\linespread{1,3}							% Linjeafstand

% ¤¤ Litteraturlisten ¤¤ %
\bibpunct[,]{(}{)}{;}{a}{,}{,} 				% Definerer parametre ved Harvard-henvisning (bl.a. parantestype og seperatortegn)
\bibliographystyle{0_bibtex/harvard}		% Udseende af litteraturlisten (Harvard-metoden - skift til fx 'plain' for tal)

% ¤¤ Dybde af overskrifter ¤¤ %
\setsecnumdepth{subsubsection}		 			% Dybden af nummerede overkrifter (part/chapter/section/subsection)
\settocdepth{subsection} 					% Dybden af overskrifter vist i indholdsfortegnelsen

% ¤¤ Lister ¤¤ %
\setlist{
  topsep=0pt,								% Vertikal afstand mellem tekst og listen
  itemsep=-1ex,								% Vertikal afstand mellem items
} 

% ¤¤ Visuelle referencer ¤¤ %
\usepackage[colorlinks]{hyperref}			% Danner klikbare referencer (hyperlinks) i dokumentet
\hypersetup{colorlinks = true,				% Opsaetning af farvede hyperlinks (interne links, citeringer og URL)
    linkcolor = black,
    citecolor = black,
    urlcolor = black
}

% ¤¤ Opsaetning af figur- og tabeltekst ¤¤ %
\captionnamefont{\small\bfseries\itshape}	% Opsaetning af tekstdelen ('Figur' eller 'Tabel')
\captiontitlefont{\small}					% Opsaetning af nummerering
\captiondelim{. }							% Seperator mellem nummerering og figurtekst
\captionstyle{\centering}					% Justering/placering af figurteksten (centreret = \centering, venstrejusteret = \raggedright)
\captionwidth{\linewidth}					% Bredden af figurteksten
\hangcaption								% Venstrejusterer fler-linjers figurtekst under hinanden
\setlength{\belowcaptionskip}{0pt}			% Afstand under figurteksten
		
% ¤¤ Opsaetning af listings ¤¤ %
\renewcommand{\lstlistingname}{Snippet} % Listing -> Snippet
\definecolor{commentGreen}{RGB}{34,139,24}
\definecolor{stringPurple}{RGB}{208,76,239}
\definecolor{dkgreen}{rgb}{0,0.6,0}
\definecolor{gray}{rgb}{0.5,0.5,0.5}
\definecolor{mauve}{rgb}{0.58,0,0.82}
\definecolor{light-gray}{gray}{0.25}
\definecolor{lighter-gray}{RGB}{240,240,240}

\lstset{language=C,					% Sprog
	basicstyle=\ttfamily\scriptsize,		% Opsaetning af teksten
	keywords={for,if,while,else,elseif,		% Noegleord at fremhaeve
			  end,break,return,case,
			  switch,function,free},
	keywordstyle=\color{blue},				% Opsaetning af noegleord
	commentstyle=\color{commentGreen},		% Opsaetning af kommentarer
	stringstyle=\color{stringPurple},		% Opsaetning af strenge
	showstringspaces=false,					% Mellemrum i strenge enten vist eller blanke
	numbers=left, numberstyle=\tiny,		% Linjenumre
	extendedchars=true, 					% Tillader specielle karakterer
	columns=flexible,						% Kolonnejustering
	breaklines, breakatwhitespace=true,		% Bryd lange linjer
}

\def\bluecolorifnotalreadymauve{%
    \extractcolorspec{.}\currentcolor
    \extractcolorspec{mauve}\stringcolor
    \ifx\currentcolor\stringcolor\else
        \color{blue}%
    \fi
}
\lstdefinelanguage{JavaScript}{
  keywords={break, case, catch, continue, debugger, default, delete, do, else, false, finally, for, function, if, in, instanceof, new, null, return, switch, this, throw, true, try, typeof, var, void, while, with},
  morecomment=[l]{//},
  morecomment=[s]{/*}{*/},
    moredelim=[s][\color{black}]{\$\{}{\}}, % same as morestring in this case
    morestring=[d]{\\"},
    morestring=**[d]{"},
    morestring=**[d]{`},
  ndkeywords={class, export, boolean, throw, implements, import, this},
  keywordstyle=\color{blue}\bfseries,
  identifierstyle=\color{black},
  commentstyle=\color{dkgreen}\ttfamily,
  stringstyle=\color{mauve}\ttfamily,
  keywordstyle=\bluecolorifnotalreadymauve,
  sensitive=true
}
\lstdefinelanguage{HTML}{
    sensitive=true,
    keywords={%
    % JavaScript
    typeof, new, true, false, catch, function, return, null, catch, switch, var, if, in, while, do, else, case, break,
    % HTML
    html, title, meta, style, head, body, script, canvas,
    % CSS
    border:, transform:, -moz-transform:, transition-duration:, transition-property:,
    transition-timing-function:
    },
    % http://texblog.org/tag/otherkeywords/
    otherkeywords={<, >, \/},   
    ndkeywords={class, export, boolean, throw, implements, import, this},   
    comment=[l]{//},
    % morecomment=[s][keywordstyle]{<}{>},  
    morecomment=[s]{/*}{*/},
    morecomment=[s]{<!}{>},
    morestring=[b]',
    morestring=[b]",    
    alsoletter={-},
    alsodigit={:}
}
\lstset{
   language=JavaScript,
   backgroundcolor=\color{lighter-gray},
   extendedchars=true,
   basicstyle=\footnotesize\ttfamily,
   showstringspaces=false,
   showspaces=false,
   numbers=left,
   numberstyle=\footnotesize,
   numbersep=9pt,
   tabsize=2,
   breaklines=true,
   showtabs=false,
   captionpos=b
}

% ¤¤ Navngivning ¤¤ %
\addto\captionsdanish{
	\renewcommand\contentsname{Indholdsfortegnelse}			% Skriver 'Indholdsfortegnelse' i stedet for 'Indhold'
	\renewcommand\appendixname{Appendiks}					% Skriver 'Appendiks' i stedet for 'Appendix'
	\renewcommand\appendixpagename{Appendiks}
	\renewcommand\appendixtocname{Appendiks}
	\renewcommand\cftchaptername{\chaptername~}				% Skriver 'Kapitel' foran kapitlerne i indholdsfortegnelsen
	\renewcommand\cftappendixname{\appendixname~}			% Skriver 'Appendiks' foran appendiks i indholdsfortegnelsen
}

% ¤¤ Kapiteludssende ¤¤ %
\definecolor{numbercolor}{gray}{0.7}		% Definerer en farve til brug til kapiteludseende
\newif\ifchapternonum

\makechapterstyle{jenor}{					% Definerer kapiteludseende frem til ...
  \renewcommand\beforechapskip{0pt}
  \renewcommand\printchaptername{}
  \renewcommand\printchapternum{}
  \renewcommand\printchapternonum{\chapternonumtrue}
  \renewcommand\chaptitlefont{\fontfamily{pbk}\fontseries{db}\fontshape{n}\fontsize{25}{35}\selectfont\raggedleft}
  \renewcommand\chapnumfont{\fontfamily{pbk}\fontseries{m}\fontshape{n}\fontsize{1in}{0in}\selectfont\color{numbercolor}}
  \renewcommand\printchaptertitle[1]{%
    \noindent
    \ifchapternonum
    \begin{tabularx}{\textwidth}{X}
    {\let\\\newline\chaptitlefont ##1\par} 
    \end{tabularx}
    \par\vskip-2.5mm\hrule
    \else
    \begin{tabularx}{\textwidth}{Xl}
    {\parbox[b]{\linewidth}{\chaptitlefont ##1}} & \raisebox{-15pt}{\chapnumfont \thechapter}
    \end{tabularx}
    \par\vskip2mm\hrule
    \fi
  }
}											% ... her

\usepackage[T1]{fontenc}
\usepackage{titlesec, blindtext, color}
\definecolor{gray75}{gray}{0.75}
\newcommand{\hsp}{\hspace{20pt}}
\titleformat{\chapter}[hang]{\Huge\bfseries}{\thechapter\hsp\textcolor{gray75}{|}\hsp}{0pt}{\Huge\bfseries}

\chapterstyle{titlesec}						% Valg af kapiteludseende - Google 'memoir chapter styles' for alternativer

% ¤¤ Sidehoved/sidefod ¤¤ %

\makepagestyle{Uni}							                                    % Definerer sidehoved og sidefod udseende frem til ...
\makepsmarks{Uni}{
	\createmark{chapter}{left}{shownumber}{}{. \ }
	\createmark{section}{right}{shownumber}{}{. \ }
	\createplainmark{toc}{both}{\contentsname}
	\createplainmark{lof}{both}{\listfigurename}
	\createplainmark{lot}{both}{\listtablename}
	\createplainmark{bib}{both}{\bibname}
	\createplainmark{index}{both}{\indexname}
	\createplainmark{glossary}{both}{\glossaryname}
}
\nouppercaseheads											                               % Ingen Caps oenskes

\makeevenhead{Uni}{\rightmark}{}{\leftmark}				                                   % Lige siders sidehoved (\makeevenhead{Navn}{Venstre}{Center}{Hoejre})
\makeoddhead{Uni}{\rightmark}{}{\leftmark}		                                           % Ulige siders sidehoved (\makeoddhead{Navn}{Venstre}{Center}{Hoejre})
\makeevenfoot{Uni}{Group \sidefod}{}{Page \textbf{\thepage}  of \textbf{\pageref{litlist}}}  % Lige siders sidefod (\makeevenfoot{Navn}{Venstre}{Center}{Hoejre})
\makeoddfoot{Uni}{Group \sidefod}{}{Page \textbf{\thepage}  of \textbf{\pageref{litlist}}}   % Ulige siders sidefod (\makeoddfoot{Navn}{Venstre}{Center}{Hoejre})
\makeheadrule{Uni}{\textwidth}{0.5pt}   					                               % Tilfoejer en streg under sidehovedets indhold
\makefootrule{Uni}{\textwidth}{0.5pt}{1mm}   				                               % Tilfoejer en streg under sidefodens indhold

\copypagestyle{Unichap}{Uni}								                               % Der dannes en ny style til kapitelsider
\makeoddhead{Unichap}{}{}{}									                               % Sidehoved defineres som blank på kapitelsider
\makeevenhead{Unichap}{}{}{}
\makeheadrule{Unichap}{\textwidth}{0pt}
															
\pagestyle{Uni}												                               % Valg af sidehoved og sidefod (benyt 'plain' for ingen sidehoved/fod)


%%%% EGNE KOMMANDOER %%%%

% ¤¤ Billede hack ¤¤ %										% Indsaet figurer nemt med \figur{Stoerrelse}{Fil}{Figurtekst}{Label}
\newcommand{\figur}[4]{
		\begin{figure}[H] \centering
			\includegraphics[width=#1\textwidth]{0_billeder/#2}
			\caption{#3}
			\label{#4}
		\end{figure} 
}

% ¤¤ Specielle tegn ¤¤ %
\newcommand{\dec}{^{\circ}}									% '\dec' returnerer et gradtegn (husk $$ udenfor aligns)
\newcommand{\decC}{^{\circ}\text{C}}						% '\decC' returnerer et gradtegn + 'C' (husk $$ udenfor aligns)
\newcommand{\m}{\cdot}										% '\m' returnerer et gangetegn


%%%% ORDDELING %%%%

\hyphenation{In-te-res-se e-le-ment}