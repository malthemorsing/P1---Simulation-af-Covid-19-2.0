\chapter{Introduction} \label{chap:introduction}

In late 2019 the world was introduced to a new strand of the corona virus, Sars-CoV-2. This new virus first appeared in Wuhan in China. From there, it moved fast across all continents and soon, every country reported their first case of COVID-19.

This project sets out to simulate the spread of COVID-19 with the eventual aim being comparing simulations of non-mitigated and mitigated spread of the illness. The theoretical foundation for the project is based in the nature of COVID-19 (virology, societal aspect of spreading, preventive measures and, of course, contact tracing).

For this project, we have decided to focus on how contact tracing affects the spread of COVID-19. In short, our problem pertains to whether contact tracing can be quantified as a variable for prevention of transmission of COVID-19. We will also focus on whether contact tracing actually works as an effective means to contain spread as well as how it compares to other preventive measures. 

We will introduce different simulation types, so we can determine the optimal simulation type for our project. With this in mind, we can also produce some simulations with and without contact tracing, which will determine its effectiveness in dealing with the spread of COVID-19.

\section{What is COVID-19?}
COVID-19 is a new disease that is thought to have occurred in Wuhan, China in December 2019 \citep{gorbalenya_severe_2020}. It is a respiratory infection in the coronaviridae family. When you are infected with COVID-19, you might experience symptoms like fever, dry cough, fatigue and worsened sense of taste and smell \citep{ssi_statens_nodate}.

With regard to testing for COVID-19, there are currently two options. The first option shows if you have a current infection, and it is done by inoculation in the mouth. The second option is an antibody test, which might tell if you have been infected before \citep{cdc_coronavirus_2020}.

On March 11, 2020, the World Health Organization declared COVID-19 a pandemic. According to www.ourworldindata.org, 1.21 million people have died from COVID-19 worldwide as of November 2, 2020 \citep{ritchie_coronavirus_2020}. As such, prevention of spread of COVID-19 is obviously paramount to the safety of both individual citizens and governments.

Individual action is essential to prevent and mitigate spread of an illness like COVID-19 that can quickly run rampant in a given population. Preventive measures will be further discussed in section \ref{sec: preventive measure} - Preventive Measures.

Different terminology is used when discussing the Coronavirus. COVID-19 is used to describe the pandemic in general, i.e. the disease that you get if you get infected. SARS-CoV-2 is used to describe the virus itself, i.e. the thing that gets you infected. For the sake of readability, only COVID-19 and SARS-CoV-2 will be used for this report.



