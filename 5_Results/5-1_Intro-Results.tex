\chapter{Results and Analyses} \label{chap:result}

This chapter is dedicated to not only give an overview of results of the simulation, but also to analyse the given results within certain parameters. This chapter is, in essence, where the problem statement (in section \vref{sec: Problem Statement}) can finally be answered.

The results shown and analysed are all based on the same seed (the seed being 808420), making it possible to reproduce the results both with and without contact tracing. 

We have made our results by simulating 1000 drops, with and without contact tracing enabled. In the following sections the two simulations will be compared, to each other and also to the real world. This is done by plotting the data found in Appendix \ref{Appendix:Data} onto graphs, these then visualises the differences and can be analysed. 

%"" This technology has been effective in almost all disease outbreaks, and as a result of our project, once again contact tracing will prove very effective against outbreaks, both national and regional.  "" - SKAL KOMME EFTER RESULTATERNE, FOR AT BEKRÆFTE VORES TEORI, AT CT VIRKER.  