\begin{lstlisting}[language=c, caption={Our simulations written in c}, captionpos=b, label={snippet:LABELNAVN}]
#include <stdio.h>
#include <stdlib.h>
#include <string.h>
#include <stdbool.h>
#include <time.h>
#include <math.h>

/*Defines the size of the population, which will be used in the simulation
 - NOTE: population cant exceed RAND_MAX, since we use this for selection when infecting, 
 which means that only the people with index 0 to RAND_MAX in the population array can be selected*/
#define populationSize 12500

/*Seed used to seed the randomizer.
 - NOTE: time(0) can be used to generate a random number on runtime, but any int can be used*/
#define seed time(0)

/*Defines the number of contacts that a person has in a given day. 
- NOTE: It will not affect the infectionrate, but will affect the number of people quarantined if contact tracing is on*/
#define avgContacts 10

#define quarantinePeriod 14 /*Defines the time traced contacts will be quarantined*/
#define timeFrame 720       /*Timeframe is the number of days the simulation will run for*/
#define numOfDrops 100      /*The number of simulations that are run when the program starts*/

#define reproductiveNumber 2.5                               /*The reproductive number is the number of people an infected person infects on average.*/
#define avgInfectionPeriod 10                                /*Defines how long people will be infected for*/
#define incubationPeriod (rand() % 8 + 3)                    /*Defines how long people will be in incubation before they become sick and can infect others*/
#define asymptomaticPeriod (rand() % avgInfectionPeriod + 1) /*Defines how long the person will be asymptomatic -NOTE: This affects contact tracing*/

#define startInfectedAmount 1 /* Defines the amount of people to be infected on day 1 of the simulation*/

#define contactTracing true                                    /*Switch to false to turn off contact tracing*/
#define contactTracingStartTime 30                             /*Defines from what day the contact tracing begins in the simulation*/
#define traceableContacts (rand() % (avgContacts * 2 - 4) + 5) /* Defines how many contacts of an infected person that can be traced */

/*The different sex a person can be*/
enum sex
{
    female,
    male
};

/*The different states that a person can be in*/
enum states
{
    susceptible,
    infected,
    recovered,
    deceased
};

/*The different age-groups a person can be in*/
enum age
{
    underSixty,
    sixties,
    seventies,
    eighties,
    ninetyPlus,
};

/*Struct which contains all variables for each person in population */
struct Person
{
    int ID;
    int age;
    int sex;
    int state;
    int asymptomaticTimer; /*The asymptomatic timer is for asymptomatic patients as well as pre-symptomatic patients*/
    int incubationTimer;
    int infectionTimer;
    int quarantineTimer;
    int infectionPeriod;
    int numOfContacts;
    bool incubation, severelyInfected, quarantine;
    int contactID[avgContacts], tracingID[avgContacts * 2];
};

typedef struct Person Person;

/* Function Prototypes */
void UpdatePopulation(Person *population, int time);
void UpdateContacts(Person *population, int infectedID);
bool CheckContacts(Person infectedPerson, Person contact, int ID);
void SaveResultsToFile(FILE *file, int results[][7], int maxValueInDrop[][4]);
void SaveDrop(Person *population, int time, FILE *file, int drop, int selectedDrop);
void SaveResults(Person *population, int results[][7], int maxValueInDrop[][4], int drop, int time);
void ResetContacts(Person *population, int time);
void setAgeSex(Person *population);
int CountState(Person *population, int status);
int CountAge(Person *population, int ageGroup);

int main()
{
    printf("Simulating...\nResults are being generated: \n");
    int startInfection;
    int localSeed = seed;

    /* file pointer is initialized and a .csv file is generated to store the data from the simulation*/
    FILE *fPointer;
    char strSeed[40];
    sprintf(strSeed, "%i", localSeed);
    char *fileName = strcat(strSeed, "seed.csv");
    fPointer = fopen(fileName, "w");
    fprintf(fPointer, "seed:;%i;Number of drops:;%i\n\n\n", localSeed, numOfDrops);

    int resultArray[timeFrame][7] = {{0}}; /*Array which contains the sum of the results for all of the drops*/
    int maxValues[numOfDrops][4] = {{0}};  /*Array which contain all of the max values we are interested in extracting from each drop*/

    for (int drops = 1; drops <= numOfDrops; drops++)
    {
        /*We seed the randomizer for each drop, so that comparison is posibble, since if its not done, the next drops are affected by the previous drops rands*/
        srand(localSeed + drops);

        /*Dynamically allocate memory according to how big the population is. 
        --- WARNING --- Usage of static allocated arrays will result in not being able to use a population exceeding 12000*/
        Person *population = calloc(populationSize, sizeof(Person));

        /* Initializes all needed variables for the entire population*/
        for (int i = 0; i < populationSize; i++)
        {
            (population + i)->ID = i;
            (population + i)->infectionTimer = avgInfectionPeriod;
            (population + i)->incubationTimer = 7;
            (population + i)->infectionPeriod = avgInfectionPeriod;
            (population + i)->asymptomaticTimer = -1;
            (population + i)->state = susceptible;
            (population + i)->incubation = false;
            (population + i)->quarantine = false;
            (population + i)->quarantineTimer = quarantinePeriod;
            (population + i)->severelyInfected = false;
            (population + i)->numOfContacts = avgContacts - 1;

            /*All of the Contact IDs and Tracing IDs are initialized to -1, to be out of bounds for the population*/
            for (int j = 0; j < avgContacts; j++)
            {
                (population + i)->contactID[j] = -1;
            }
            for (int k = 0; k < avgContacts * 2; k++)
            {
                (population + i)->tracingID[k] = -1;
            }
        }

        /*Function call to set the age and sex distribution of the population*/
        setAgeSex(population);

        printf("%i ", drops);

        /* Infects the first people so the simulation can start*/
        for (int i = 0; i < startInfectedAmount; i++)
        {
            startInfection = rand() % populationSize;
            (population + startInfection)->state = infected;
            (population + startInfection)->asymptomaticTimer = asymptomaticPeriod;
        }

        /* For every day in the timeframe, we check for an update of the state for each person in the population*/
        for (int days = 1; days <= timeFrame; days++)
        {
            UpdatePopulation(population, days);
            ResetContacts(population, days);
            SaveResults(population, resultArray, maxValues, drops, days);
        }

        free(population); /*The allocated memory gets freed up after we're done using it*/
    }

    SaveResultsToFile(fPointer, resultArray, maxValues);
    fclose(fPointer); /*The file is closed and data is saved*/
    printf("\nThe simulation has finished, a data file has been created\n");
    return EXIT_SUCCESS;
}

/*Function which saves the sum of the results of all the drops in the result array, 
it also saves the max value for the number of infected and hospitalized in each drop in the max value array*/
void SaveResults(Person *population, int results[][7], int maxValueInDrop[][4], int drop, int time)
{

    /*Temporary array which stores the time, number of people in different states and som sub-states.
    Its used to load the values in the result array*/
    int tempResult[7] = {0};

    tempResult[0] = time;

    tempResult[1] = CountState(population, susceptible);
    tempResult[2] = CountState(population, infected);
    tempResult[3] = CountState(population, recovered);
    tempResult[4] = CountState(population, deceased);

    for (int index = 0; index < populationSize; index++)
    {
        if ((population + index)->quarantine == true)
        {
            tempResult[6]++;
        }
        if ((population + index)->severelyInfected == true)
        {
            tempResult[5]++;
        }
    }

    if (maxValueInDrop[drop - 1][1] < CountState(population, infected))
    {
        maxValueInDrop[drop - 1][0] = time;
        maxValueInDrop[drop - 1][1] = CountState(population, infected);
    }
    if (maxValueInDrop[drop - 1][3] < tempResult[5])
    {
        maxValueInDrop[drop - 1][2] = time;
        maxValueInDrop[drop - 1][3] = tempResult[5];
    }

    /*the results are added to the resul array*/
    results[time - 1][0] = tempResult[0];

    for (int i = 1; i < 7; i++)
    {
        results[time - 1][i] += tempResult[i];
    }
}

/* Updates the state of each person, this will be done accordingly to where they are in their incubation/infection period */
void UpdatePopulation(Person *population, int time)
{
    int tracingNum;

    /*Motality rate for the different age groups, these numbers were calculated from the official danish COVID-19 statistics on 09-12-2020.
    - NOTE: These numbers were calculated by the number of deceased in age groups devided by the number of hospitalized in the same age groups.
    This is because it is assumed in this simulation that the deceased were hospitalized before they died of the illness*/
    static double mortalityRateByAge[5] = {0.0132, 0.8923, 0.196, 0.3453, 0.7346};

    /* Checks through the entire population to see if the different checks or timers apply for each person*/
    for (int i = 0; i < populationSize; i++)
    {
        /*Checks if the individual fulfills the criteria for dying due to covid*/
        if ((population + i)->state != deceased && (population + i)->severelyInfected == true && (population + i)->incubation == false &&
            ((double)rand() / (double)RAND_MAX) <= mortalityRateByAge[(population + i)->age] / avgInfectionPeriod)
        {
            (population + i)->state = deceased;
            (population + i)->asymptomaticTimer = -1;
            (population + i)->infectionTimer = -1;
            (population + i)->severelyInfected = false;
        }
        /* Makes sure that the incubation timer ticks down if a person is in incubation*/
        if ((population + i)->incubation == true)
        {
            (population + i)->incubationTimer--;

            /* Makes sure that a person is taken out of incubation if incubation timer reaches 0*/
            if ((population + i)->incubationTimer == 0)
            {
                (population + i)->incubation = false;
                (population + i)->incubationTimer = -1;
            }
        }

        /* Makes sure that the quarantine timer ticks down if a person is in quarantine*/
        if ((population + i)->quarantine == true)
        {
            (population + i)->quarantineTimer--;

            /* Makes sure that a person is taken out of quarantine if the quarantine timer reaches 0*/
            if ((population + i)->quarantineTimer == 0)
            {
                (population + i)->quarantine = false;
                (population + i)->quarantineTimer = quarantinePeriod;
            }
        }

        /* Makes sure that a person get the state recovered if their infection timer reaches 0*/
        if ((population + i)->infectionTimer == 0)
        {
            (population + i)->state = recovered;
            (population + i)->severelyInfected = false;
            (population + i)->infectionTimer = -1;
        }
        /* Makes sure that if a person gets covid symptoms they get put in quarantine and their tracable contacts 
           gets traces and put in quarantine aswell*/
        tracingNum = traceableContacts;

        if ((population + i)->asymptomaticTimer == 0 && (population + i)->state == infected &&
            (population + i)->quarantine == false && contactTracing == true && time >= contactTracingStartTime)
        {
            (population + i)->quarantine = true;

            for (int j = 0; j < tracingNum; j++)
            {
                if ((population + i)->tracingID[j] >= 0)
                {
                    (population + (population + i)->tracingID[j])->quarantine = true;
                }
            }
        }

        /* Makes sure that if the asymptomatic timer reaches 0, the person is no longer symptomatic. 
        Setting the timer to -1 ensures that they as well as their contacts dont end up in quarantine multiple times because of contact tracing*/
        if ((population + i)->asymptomaticTimer == 0)
        {
            (population + i)->asymptomaticTimer = -1;
        }
        /* If a person is infected and able to infect others, their contacts are updated*/
        if ((population + i)->state == infected && (population + i)->incubation == false && (population + i)->quarantine == false)
        {
            UpdateContacts(population, i);
        }
        /* If a person is infected at is out of incubation period, their asymptomatic timer and infection timer is ticked down*/
        if ((population + i)->state == infected && (population + i)->incubation == false)
        {
            (population + i)->infectionTimer--;
            (population + i)->asymptomaticTimer--;
        }
    }
}

/* Genereates the contacts for the infected, and chooses whether the contact will be infected or not */
void UpdateContacts(Person *population, int infectedID)
{
    /* The used variables are defined*/
    static double severityAgeFactor[5] = {0.0266, 0.1255, 0.28874, 0.4707, 0.41935};
    double infectionChance;
    int contact;
    int neededContacts = (population + infectedID)->numOfContacts;

    /* Goes through every contact that the infected has*/
    for (int i = 0; i < neededContacts; i++)
    {
        /* Picks a random contact for the person as long as they haven't gotten number of contacts they need already.
        This loop also makes sure the contact is valid(i.e. not in quarantine, deceased, etc.)*/
        do
        {
            contact = (rand() % populationSize);
        } while (CheckContacts(*(population + infectedID), *(population + contact), contact));

        /* Sets the infected person as a contact of the contact and counts down their number of contacts*/
        (population + contact)->contactID[(population + contact)->numOfContacts] = (population + infectedID)->ID;
        (population + contact)->numOfContacts--;

        /* Sets the contact as a contact of the infected person and counts down their number of contacts*/
        (population + infectedID)->contactID[(population + infectedID)->numOfContacts] = (population + contact)->ID;
        (population + infectedID)->numOfContacts--;

        /* Sets a chance of infection*/
        infectionChance = (double)rand() / (double)RAND_MAX;

        /* Checks if the contact will get infected*/
        if (infectionChance < (reproductiveNumber / avgContacts / (population + infectedID)->infectionPeriod))
        {
            /* IF the contact is susceptible they will get infected*/
            if ((population + contact)->state == susceptible)
            {
                (population + contact)->state = infected;
                (population + contact)->incubation = true;
                (population + contact)->incubationTimer = incubationPeriod;
                (population + contact)->asymptomaticTimer = asymptomaticPeriod;

                /* Checks if the contact will get severely infected*/
                if ((double)rand() / (double)RAND_MAX <= severityAgeFactor[(population + contact)->age])
                {
                    (population + contact)->severelyInfected = true;
                    (population + contact)->asymptomaticTimer = 2; /*assumption that severe infections have a short asymptomatic period.*/
                }
            }
            else /* If the contact doesn't get infected, the check shall then continue to the next contact*/
            {
                continue;
            }
        }
    }
}

/* Function that takes in the results of the simulation prints it in the file*/
void SaveResultsToFile(FILE *file, int results[][7], int maxValueInDrop[][4])
{
    fprintf(file, "Drop nr.;Day;Max infected in drop;;Day;Max hospitalized in drop\n");

    for (int j = 0; j < numOfDrops; j++)
    {
        fprintf(file, "%i;", j + 1);
        fprintf(file, "%i;", maxValueInDrop[j][0]);
        fprintf(file, "%i;;", maxValueInDrop[j][1]);
        fprintf(file, "%i;", maxValueInDrop[j][2]);
        fprintf(file, "%i\n", maxValueInDrop[j][3]);
    }

    fprintf(file, "\n\n\n");

    fprintf(file, "Time;Susceptible;Infected;Recovered;Deceased;Hospitalized;Quarantined\n");
    fprintf(file, "0;%i;%i;0;0;0;0\n", populationSize - startInfectedAmount, startInfectedAmount);
    for (int i = 0; i < timeFrame; i++)
    {
        /*The result array is the sum of all the numbers from all of the drops on all days. This prints the average for each result for all of the days*/
        fprintf(file, "%i;%lf;%lf;%lf;%lf;%lf;%lf\n", results[i][0], ((double)(results[i][1] / (double)numOfDrops)), (((double)results[i][2] / (double)numOfDrops)),
                (((double)results[i][3] / (double)numOfDrops)), (((double)results[i][4] / (double)numOfDrops)), (((double)results[i][5] / (double)numOfDrops)),
                (((double)results[i][6] / (double)numOfDrops)));
    }
}

/* Bool function to assure that the infected's selected contact is valid*/
bool CheckContacts(Person infectedPerson, Person contact, int ID)
{
    for (int i = 0; i < avgContacts; i++)
    {
        if (infectedPerson.contactID[i] == ID || infectedPerson.numOfContacts < 0 || contact.numOfContacts < 0 ||
            infectedPerson.ID == ID || contact.quarantine == true || contact.state == deceased)
        {
            return true; /*Returns true, means that a new contact needs to be found*/
        }
    }
    return false; /*Returns false, means that the checked contact meets all of the criteria for being selected*/
}

/* Function that counts the amount of people in each state*/
int CountState(Person *population, int status)
{
    int i = 0;
    for (int j = 0; j < populationSize; j++)
    {
        if ((population + j)->state == status)
        {
            i++;
        }
    }
    return i;
}

/* Function that resets the contact array and puts old contact id's into the tracing array*/
void ResetContacts(Person *population, int time)
{
    for (int i = 0; i < populationSize; i++)
    {
        for (int k = 0; k < avgContacts; k++)
        {
            if (time % 2 == 0)
            {
                (population + i)->tracingID[k] = (population + i)->contactID[k];
            }
            else
            {
                (population + i)->tracingID[k + avgContacts] = (population + i)->contactID[k];
            }
        }

        (population + i)->numOfContacts = avgContacts - 1;
        for (int j = 0; j < avgContacts; j++)
        {
            (population + i)->contactID[j] = -1;
        }
    }
}

/*Age set per person, dependent on sex.*/
void setAgeSex(Person *population)
{
    /*Establishes the percentage of males in different age groups*/
    double underSixtyPercentageMale = 0.3751;
    double sixtiesPercentageMale = 0.0573;
    double seventeesPercentageMale = 0.0457;
    double eigthiesPercentageMale = 0.0157;
    double ninetyPlusPercentageMale = 0.0024;

    double totalMales = underSixtyPercentageMale + sixtiesPercentageMale + seventeesPercentageMale + eigthiesPercentageMale + ninetyPlusPercentageMale;

    /*Defines to establish the percentage of females in different age groups
 - NOTE: that ninetyPlus for females is not needed since they make up the remaining part of population*/
    double underSixtyPercentageFemale = 0.3664;
    double sixtiesPercentageFemale = 0.0584;
    double seventeesPercentageFemale = 0.0505;
    double eigthiesPercentageFemale = 0.0225;

    /*For males*/
    for (int m = 0; m < ceil(populationSize * totalMales); m++)
    {
        (population + m)->sex = male;

        if (m < ceil(populationSize * (underSixtyPercentageMale)))
        {
            (population + m)->age = underSixty;
        }
        else if (m < ceil(populationSize * (underSixtyPercentageMale + sixtiesPercentageMale)))
        {
            (population + m)->age = sixties;
        }
        else if (m < ceil(populationSize * (underSixtyPercentageMale + sixtiesPercentageMale + seventeesPercentageMale)))
        {
            (population + m)->age = seventies;
        }
        else if (m < ceil(populationSize * (underSixtyPercentageMale + sixtiesPercentageMale + seventeesPercentageMale + eigthiesPercentageMale)))
        {
            (population + m)->age = eighties;
        }
        else if (m < ceil(populationSize * (totalMales)))
        {
            (population + m)->age = ninetyPlus;
        }
    }

    /*For females*/
    for (int f = ceil(populationSize * totalMales) + 1; f < populationSize; f++)
    {
        (population + f)->sex = female;

        if (f < ceil(populationSize * totalMales) + floor(populationSize * underSixtyPercentageFemale))
        {
            (population + f)->age = underSixty;
        }
        else if (f < ceil(populationSize * totalMales) + floor(populationSize * (underSixtyPercentageFemale + sixtiesPercentageFemale)))
        {
            (population + f)->age = sixties;
        }
        else if (f < ceil(populationSize * totalMales) + floor(populationSize * (underSixtyPercentageFemale + sixtiesPercentageFemale + seventeesPercentageFemale)))
        {
            (population + f)->age = seventies;
        }
        else if (f < ceil(populationSize * totalMales) + floor(populationSize * (underSixtyPercentageFemale + sixtiesPercentageFemale + seventeesPercentageFemale + eigthiesPercentageFemale)))
        {
            (population + f)->age = eighties;
        }
        else if (f < populationSize)
        {
            (population + f)->age = ninetyPlus;
        }
    }
}

/* Function that counts the amount of people in each agegroup*/
int CountAge(Person *population, int ageGroup)
{
    int i = 0;
    for (int j = 0; j < populationSize; j++)
    {
        if ((population + j)->age == ageGroup)
        {
            i++;
        }
    }
    return i;
}

/*Function which can save an individual drop to that data file, in case further analysis of a specific drop is neccesary*/
void SaveDrop(Person *population, int time, FILE *file, int drop, int selectedDrop)
{
    int q = 0, a = 0, s = 0;
    if (drop == selectedDrop)
    {
        for (int index = 0; index < populationSize; index++)
        {
            if ((population + index)->quarantine == true)
            {
                q++;
            }
            if ((population + index)->severelyInfected == true)
            {
                s++;
            }
            if ((population + index)->asymptomaticTimer >= 0)
            {
                a++;
            }
        }
        fprintf(file, "%i;%i;%i;%i;%i;%i;%i;%i\n", time, CountState(population, susceptible), CountState(population, infected),
                CountState(population, recovered), CountState(population, deceased), s, q, a);
    }
}
\end{lstlisting}