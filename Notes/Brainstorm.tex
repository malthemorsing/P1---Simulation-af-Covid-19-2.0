\subsection{Brainstorm} 

\subsubsection{Bud fra Malthe}
Udvælg et lukket, ikke-offentligt område for at gennemgå smitte inden for disse rammer. 
Tag udgangspunkt i disse data for at finde elementer, der skal inddrages i simuleringen.


\subsubsection{Bud fra Daniel (?)}
Tag kontakt til Smitte|stop el.lign. myndighed/ekspert (f.eks Statens Serum Institut) og hør dem om vi kan få adgang til deres data.
Enten byg/find et framework som kan tilpasses vores ønsker ift. simulering. 


\subsubsection{Bud fra Astrid}
Vinkel:
Kritisk blik på de elementer og huskeregler, der anvendes til smittesporing og kontaktregistrering på nuværende tidspunkt.


Denne vinkel vil behøve en gennemgang af COVID-19 som sygdom, hvilket kan blive tungt.


\section{spørgsmål til at komme i gang med simulering}

From this analysis, we plucked out the questions which we found to be most socially relevant, personally interesting and workable within our field. Our focal points were therefore strongly influenced by the following questions:
\begin{itemize}
\item How can software solutions aid in preventing further spread of COVID-19?
\item How can software solutions aid in continuing daily life with little to no physical contact with others?
\item Why make use of preventive measures against COVID-19 at all?
\end{itemize}