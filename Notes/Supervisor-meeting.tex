\section{First meeting - 9/10-20}
The proposal is very vague, so there are many opportunities

Google if someone else have done something similar (they have /D) 

Start with researching basic information regarding COVID spread 

Simulate the effect of spreading COVID without any restriction measures
- what happens if you include restrictions? (masks, working from home, changing key variables etc)

\subsubsection{Aspects to think about:}
\begin{itemize}
    \item probability of the people to run into each other
    \item What is the effect on society?
        \item hospital capacity
        \item cost
\end{itemize}

\subsubsection{Focus on:}
Model to calculate spread of COVID
- Divide population into groups (different probability of getting the virus)

- Include variables (masks, social distancing, etc) that can influence the model's output
- Be realistic regarding risk of infection

Schedule up until January
- Gantt chart

Scenario:
Utilize publically available data to build the simulation from as realistic numbers as possible
If able to, ask Aalborg hospital for data, but if not, make some up

\subsection{For next meeting, Monday @ 11:15}
Prepare Gantt chart

Figure out variables and population division to build the model upon

\section{Second meeting - 12/10-20}
No agenda - because we're so smart

For Brainstorm:
Think about the general problem and model, wait with simulation -> Simulation shouldn't start until after statusseminar
- THINK BIG!

Follow available framework
- Thinking solution up from scratch is also allowed, but focus on problem
- Theory first 
-- What is the problem?
--- What are the relevant aspects of the problem?
---- How would we phrase the problem?
Afterwards, we can go simpler if the problem is too complex - BUT START BIG

Focus both on spread and societal penalty of the spreading of the virus

Search for numerical evidence (SSI, hospitals, etc)
- Consider which parametres are left out/are not immediately obvious (jobs, etc)
- In short, figure out an educated guess

How to quantify:
- Focus first on parametres of spreading
- Rank the parametres according to relevance
- Argue for relevance of parametres
- Reliability of the parametres (which might be educated guesses?)

Afterwards, focus on the interdependencies of the variables and parametres:
- older people with higher income are more likely to move around more
- people who live outside of the city are less likely to move around
- hospital capacity per region/city 
- what jobs can be adapted to WFH, and what jobs cannot?
CONSIDER MAKING A VISUAL FIGURE FOR THIS

For next time: 
\begin{itemize}
\item Make agenda
\item Research
\item Do not rush, assign enough time for each point
\item Give each person enough to do
\item Rotate responsibilities
\item Send agenda and timeplan to Gilberto by Tuesday 20/10-20
\end{itemize}
\newpage



\section{Third Meeting - 21/10-20}
\subsection{Agenda: }
\begin{enumerate}
\item The purpose of the meeting
\item Group contract 

Gilberto will get back to us after the meeting on whether it's approved,
then we can sign it

\item P1 Project update

Strong advice: Make a weekly plan with what we are expected to achieve

We might lose track without it

Add weekly goals to timeplan regarding layout

\textbf{Report layout:}

Split Data-chapter and Simulation-chapter into two


Also add chapter on simulation results


Otherwise fine


Starting broad and narrowing down => Gilberto is a fan!

\item Further work for next week

Make Gantt chart - revise timeplan to include parallel workflow, email to Gilberto when done

\textbf{Gilberto says}

Brainstorm seems quite exhaustive

More sources on contact tracing and preventive measures

More sources on specific statistics - region or city
If we cannot find this kind of regional or local source, we can assume that it is the same as the general

Include sources and focus on hospital capacity => Angle towards societal problem

Gilberto is on board with getting numbers from SSI over e-mail, but not with going there physically to get the data. CC Gilberto on the e-mail to SSI.

\textbf{Two stages in the project:}

1st stage where we simulate without any preventive measures

2nd stage where we simulate with the chosen variables and scope that include preventive measures

Focus on healthcare and economy
-> If we don't utilize any preventive measures, what are the consequences?

\textbf{Hospital capacity}

Get data from Aalborg Hospitalerne about capacity <- This will be our limit for the contact tracing

\item Scheduling next week + meeting (perhaps Wed-Thursday) 

Work on problem analysis until next week, then discuss with Gilberto

\textbf{Find a model for how to simulate}

Find different levels of infection that we can simulate

Consider parametres to include in simulation

Start simple!

\textbf{Literature on simulation approaches}

Find it, consider it

Focus on event-based simulations

\textbf{Meeting}

Thursday 29th October, 13:00

- Only contact Gilberto through e-mail

\item AOB (Any other business) 

\textbf{How to include code in report:}

Include important, functional code in the report when needed

Leave the code in its entirety an appendix

\item Evaluation of the meeting 
\end{enumerate}

\section{Fourth Meeting} - 29/10-20

Mention all parametres during outline 

Select parametres for simulation based on relevance and simplicity

RESTRUCTURE

Have a Scope of report and Scope of simulation - Scope of simulation goes as introduction to Chapter 3, Scope of report goes to introduction

Include HV-map in Problem Analysis (before Variables and such). Describe in detail what is chosen for the model. - Tjek

Virology goes first, then who is affected (societal aspects) and then what we can do to limit outbreak (preventive measures)

Model should come earlier - write down on pen and paper - then implement model in C - this is where we talk about simulation - 

INSERT CHAPTER 4 - Done
For performance results - results of simulation

IN CHAPTER further research
Here goes the parametres that we could not include in the model