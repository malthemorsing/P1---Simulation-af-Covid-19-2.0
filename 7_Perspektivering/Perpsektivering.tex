\chapter{Further Research} \label{chap:Further research}

This chapter is dedicated to subjects that are related to the subject of the project as a whole, but would simply take up too much time to research further within the scope of this project. There are several interesting angles that could be explored in the peripheries of preventing COVID-19, some of which can be read below.

\section{Updated Analysis}

%If you have updated information on statistics and so on which are not the ones you have based your analysis on, you can always comment on this in the future work section, describing what an updated version of the program should consider.

%In summary, if the more “correct” assumptions does not lead to major extra work, then you should go for it. Otherwise keep the “old” assumptions, and mention the new updated analysis as future work. I recommend to discuss this in the team and make your choice asap.

\section{Biological Sex as a Factor}

As mentioned in 


\section{Preventive Measures as a Factor}

Assuming that only one preventive measure (in this case contact tracing) is used against a pandemic is unrealistic at best. Therefore, other preventive measures can be included in further research to generate results that are more in line with observable spread of COVID-19. 

There are several ways to go about this. Firstly, a simple but arguably uncertain method is to adjust the R-value (as defined in subsection \vref{subsec:Data in Sim}) according to the efficacy of the given preventive measure. If this method is used to include preventive measures, it is more likely that any results generated this way will end up being inaccurate.

A different method is to include each preventive measure as a function in the source code (Appendix \ref{Appendix: SourceCode}). This, however, would create a code that is significantly longer and more complex than the code is in its current state. Therefore, computational efficiency can be brought into question. 