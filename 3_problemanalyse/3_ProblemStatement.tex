\section{Problem Statement} \label{sec: Problem Statement}
After laying a proper, theoretical foundation, it is now possible to formulate a problem statement. As established, the finished problem statement must coincide with the project catalogue (Appendix \ref{P1Katalog}). Therefore, the centre of the problem statement must relate to contact tracing of COVID-19. Beyond this, we have wished to focus on how the spread of COVID-19 can be affected by contact tracing. This has led us to the following problem statement:

\begin{center}
    \textit{How to model and simulate the spread of COVID-19 in a given population with or without contract tracing? 
How can this simulation be used to extrapolate relevant conclusions with different parameters?}
\end{center}

The parameters mentioned are things such as the amount of susceptible, infected, recovered etc. This problem statement allows us to both collect and treat data relating to the efficiency of contact tracing. To properly quantify the effect of contact tracing, we will be making use of a "ceiling" - in effect, a maximal capacity. This max capacity will be hospital capacity for the given population. If contact tracing mitigates the spread enough to keep the rate of severely infected under the hospital capacity, we can conclude that contact tracing has enough of a positive effect on the pandemic.