\section{Contact Tracing}\label{Contact Tracing}

Contact tracing is the simple act of tracking the spread of an illness through a given population. Contact tracing is typically only used in scenarios where the illness has turned into an epidemic or pandemic - such as the case with COVID-19. In its most basic form, contact tracing can be done through word-of-mouth. In the modern world, contact tracing is typically handled by an authority with medical expertise. With new technology, contact tracing can be optimized even further.

With COVID-19 being easily spread from one person to another, it is important to identify who a person (who have tested positive) have been in contact with, within the time period that they could have spread the virus to other people. These people are called contacts, and will be observed for two weeks, to see if they develop any symptoms, and if so, if they have contracted the virus. The goal of course is to minimise the spread of COVID-19 pandemic\citep{who_headquarters_coronavirus_2020}.

The World Health Organisation have written about the several steps of contact tracing. First of all, contacts need to be defined. In short, who is considered a contact? In the case of COVID-19, the definition of a contact is ``a person who has been exposed to someone else infected with the virus that causes COVID-19, from 2 days before to 14 days after the person started to show symptoms.'' \citep{who_headquarters_coronavirus_2020}

Secondly, the defined contacts need to be identified. Here, the infected person will be interviewed,
to find out who they have had contact with \citep{who_headquarters_coronavirus_2020}. Typically, there is a special focus on contacts in the 48 hours leading up to the testing.

Thirdly, the identified contacts should be contacted to inform them. This can be done either through an app, over the phone or via e-mail to let them know that they have been contact traced and what the purpose of contact tracing is. In this context, they must also be told how their personal data is used and what it is being used for. The contact will also be informed on the symptoms to be aware of as well as how, when and where to quarantine \citep{who_headquarters_coronavirus_2020}.

Fourthly, the contact will be encouraged to stay in isolation to prevent further spreading of the disease. During quarantine, the contact should be monitored for 14 days after exposure, to see if the person develops any symptoms or gets ill. If the contact receives a positive test, they will likewise need to trace their own contacts. Lastly, the information of each contact person is stored in a database daily, updating the person's health status \citep{who_headquarters_coronavirus_2020}.

WHO adds that contact tracing does not necessarily need technological tools. The aid of digital contact tracing has optimised the process. Via digital contact tracing through usage of apps and Bluetooth, it can certainly help streamline the process of contact tracing. COVID-19 was not the first illness that modern, digital contact tracing was used for. Many solutions used for COVID-19 were originally developed to combat the spread of Ebola virus \citep{who_headquarters_coronavirus_2020}.

\subsection{Contact Tracing in Denmark} \label{Contact Tracing in Denmark}

In Denmark, citizens have the option to use the newly developed app smitte|stop for contact tracing. The app was specifically developed for contact tracing of COVID-19. With this type of digital contact tracing, a significant advantage is given each person in regards to efficiency. The app simply notifies the user if they have been in close contact with someone who has tested positive \citep{smittestop_smittestop-app_nodate}. Since a majority of people who test positive for COVID-19 are asymptomatic, this type of contact tracing can arguably minimise spread of COVID-19.

The smitte|stop app works relatively simply. It generates a random ID between users' mobile devices if two or more users have been in close proximity for more than 15 minutes in any given location. The exchange of these randomly generated IDs is done via Bluetooth. The reasoning behind the random generation is to keep each user's information separate from the app. In essence, the only information the app has access to is the device's location as well as the device's Bluetooth connection. To add further security and anonymity to the users of smitte|stop, the randomly generated IDs are updated every 15 minutes. The app itself regularly compares the collected IDs to that of the infected person. If there is a match, the app will notify you and advise you on future actions \citep{smittestopdk_download_nodate}.

Similarly, if you are the person who has tested positive, you can simply add to your app that you are now infected. The app will verify your identity through your NemID (the common, secure login solution in Denmark) to check whether you have a positive test to prevent fake positives. As a last step, smitte|stop asks your permission to share your previously generated IDs within the last 14 days. This enables the app to notify any and all people you have been in close contact with for more than 15 minutes in the last 14 days \citep{smittestopdk_download_nodate}.

To summarize, this type of digital contact tracing essentially automates and optimises the process for the population of Denmark through the app smitte|stop. An improved method for contact tracing (such as this is) is especially vital for the minimisation of spread of illnesses that present themselves asymptomatically. Since COVID-19 has proven to be largely asymptomatic, an app like smitte|stop or similar can arguably aid in prevention of spread.