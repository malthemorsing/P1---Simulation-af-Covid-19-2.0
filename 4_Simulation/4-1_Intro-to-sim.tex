\chapter{Simulation Design} \label{chap:simulation}

The chapter "Simulation" focuses on the design and execution of our simulation, herein the models created to help produce our simulation. These models and all factors surrounding them will be described in this chapter. However, to properly create and treat a simulation, a theoretical outline must first be written regarding simulations. 

Therefore, this chapter begins with basic theory regarding what a simulation is, what type of simulations might be relevant for this specific project, as well as the overall scope of the simulation. 

Afterwards, an explanation of the composition of our simulation follows. This explanation includes limitations within our simulation, and challenges posed by our source code as well as defences for the data used to output the raw data. 

\section{What is a Simulation?}
In this report on simulating COVID-19 we will use \citeauthor{choi_modeling_2013}'s academic definition of a computer simulation. They define these simulations as the following:

\say{A dictionary definition of simulation is the technique of imitating the behaviour of some situation by means of an analogous situation or apparatus to gain information more conveniently or to train personnel, while an academic definition of computer simulation is the discipline of designing a model of a system, simulating the model on a digital computer, and analyzing the execution output.} \citep{choi_modeling_2013}

Simulations make it possible for anyone to gain an overview of how specific events change and develop, typically over a period of time. These events could be related to businesses researching new methods, statistics for population growth or the spread of an illness. Therefore, a simulation is the ideal tool for predicting the spread of COVID-19. 

Simulation modeling solves real-world problems safely and efficiently. It provides an important method of analysis which is easily verified, communicated, and understood. Across industries and disciplines, simulation modeling provides valuable solutions by giving clear insights into complex systems.

Simulation enables experimentation on a valid digital representation of a system. Unlike physical modeling, such as making a scale copy of a building, simulation modeling is computer based and uses algorithms and equations. Simulation software provides a dynamic environment for the analysis of computer models while they are running, including the possibility to view them in 2D or 3D.

The uses of simulation in business are varied and it is often utilized when conducting experiments on a real system is impossible or impractical, often because of cost or time.

The ability to analyze the model as it runs sets simulation modeling apart from other methods, such as those using Excel or linear programming. By being able to inspect processes and interact with a simulation model in action, both understanding and trust are quickly built.